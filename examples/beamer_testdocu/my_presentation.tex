\documentclass{beamer}[10]
\usepackage{pgf}
\usepackage[ngerman]{babel}
\usepackage[utf8]{inputenc}
\usepackage{beamerthemesplit}
\usepackage{graphics,epsfig, subfigure}
\usepackage{url}
\usepackage{hyperref}


\setbeamercovered{transparent}
\mode<presentation>
\usetheme[numbers,totalnumber,compress,sidebarshades]{PaloAlto}
\setbeamertemplate{footline}[frame number]
\useinnertheme{circles}
\usefonttheme[onlymath]{serif}
\setbeamercovered{transparent}
\setbeamertemplate{blocks}[rounded][shadow=true]


\title{A Title}
\author{My Name is}
\institute{I work at}
\date{\today}



\begin{document}
\frame{\titlepage \vspace{-0.5cm}
}

\frame
{
\frametitle{Overview}
\tableofcontents
}


\part{Introduction}
\section{Basic Theory}
    \subsection{Foundation}
        \subsubsection{Rule 1}
                \begin{frame}{Name}{\subsubsecname}	content \end{frame}
        \subsubsection{Rule 2}
                \begin{frame}{Name}{\subsubsecname}	content \end{frame}
    \subsection{Advanced}
\section{Complex Theory}
    \subsection{Some Math}
        \subsubsection{Equation 1}
                \begin{frame}{Name}{\subsubsecname}	content \end{frame}
        \subsubsection{Equation 2}
                \begin{frame}{Name}{\subsubsecname}	content \end{frame}
    \subsection{Some Programming}

\part{Implementation}
\section{Golang Basics}
    \subsection{Data Structures}
        \subsubsection{Array}
                \begin{frame}{Name}{\subsubsecname}	content \end{frame}
        \subsubsection{Dictionary}
                \begin{frame}{Name}{\subsubsecname}	content \end{frame}
    \subsection{Concurrency}
\section{Implementation in Golang}

\end{document}